
% Thesis Abstract -----------------------------------------------------


%\begin{abstractslong}    %uncommenting this line, gives a different abstract heading
\begin{abstracts}         %this creates the heading for the abstract page

	We give an overview of the foundations and the basic results of the classical \BN Theory, dealing with the geometry of the moduli varieties parametrizing effective divisors and linear series on a given curve.\\
	We mainly follow the treatment proposed in the book \emph{Geometry of Algebraic Curves} written by Arbarello, Cornalba, Griffiths and Harris. 
	The classical theory, as presented in the book, was developed during the last century for curves over the complex numbers.\\ 
	Our discussion, instead, avoids the use of any complex-analytic tool and it is completely formulated in terms of modern algebraic geometry. As a result of this more general approach, we are able to generalize two key results of the classical theory -- the Existence and Connectedness Theorems -- to curves over an arbitrary algebraically closed field.\\
	It is important to highlight that the \BN theory heavily relies on homological algebra and, in particular, a crucial role is played by the so called Petri's map, which is in fact a cohomolgical cup-product homomorphism. Hence it seems reasonable to guess that the concepts described in the classical theory are not strictly dependent on complex analysis and, thus, that most of its results can be extended to more general fields.
	\vspace{12em}
	\begin{center}
		\footnotesize{
			The complete LaTex source of this document can be downloaded from the repository \\ 
			\href{https://github.com/AndreaBarbon/Algebraic-Brill-Noether-Theory}{https://github.com/AndreaBarbon/Algebraic-Brill-Noether-Theory}
		}
	\end{center}

\end{abstracts}
%\end{abstractlongs}


% ---------------------------------------------------------------------- 

% this file is called up by thesis.tex
% content in this file will be fed into the main document

% Glossary entries are defined with the command \nomenclature{1}{2}
% 1 = Entry name, e.g. abbreviation; 2 = Explanation
% You can place all explanations in this separate file or declare them in the middle of the text. Either way they will be collected in the glossary.

% required to print nomenclature name to page header
\markboth{\MakeUppercase{\nomname}}{\MakeUppercase{\nomname}}


% ----------------------- contents from here ------------------------

% chemicals
\nomenclature{DAPI}{4',6-diamidino-2-phenylindole; a fluorescent stain that binds strongly to DNA and serves to marks the nucleus in fluorescence microscopy} 
\nomenclature{DEPC}{diethyl-pyro-carbonate; used to remove RNA-degrading enzymes (RNAases) from water and laboratory utensils}
\nomenclature{DMSO}{dimethyl sulfoxide; organic solvent, readily passes through skin, cryoprotectant in cell culture}
\nomenclature{EDTA}{Ethylene-diamine-tetraacetic acid; a chelating (two-pronged) molecule used to sequester most divalent (or trivalent) metal ions, such as calcium (Ca$^{2+}$) and magnesium (Mg$^{2+}$), copper (Cu$^{2+}$), or iron (Fe$^{2+}$ / Fe$^{3+}$)}



